
%\begin{figure}
%\includegraphics[scale=0.2]{./1.jpg}
%\end{figure}


Intro:

This lab aims to examine the charge collection and transport properties of a pixelated CdZnTe detector. This Redlen Technologies CZT crystal (crystal number 1498, UCBCZT2) is a 1x1x1 cm3 CZT crystal segmented into nine pixels. Pixels measure 0.75x0.75 mm2 with a 1mm pitch, while the cathode is the full size of the rear face of the crystal. Each pixelated anode is connected an individual built-in pre-amplifier. As two of the pixels are currently not working, we only use the 7 anodes that are and the cathode signals. Thus, we can only use one digitizer (SIS 3302) that has a total of 8 signal inputs with USB interface (SIS 3150) to collect the signal pulses. The correspondence of anode output and port number on the digitizer will be given in the Appendix. The sampling frequency is 100MHz. An optimized trapezoidal filter will be applied to the signal to give spectra for the source signal of Am-241 or Cs-137 [CURRENTLY (11/29) using eventdata trap filter]. Experimental determination of mobility-lifetime product (mutau product) for both electrons and holes in the CZT has been determined using two methods. First is by curve fitting the Hecht relation using photopeak position with different bias voltage; while the second is by assuming constant fractional carrier loss rate per drift distance with different bias voltage. As all of the spectra gathered by illuminating anode with low-energy gammas have poor resolution, we limit our calculation to electron mu-tau product. Additionally, depth of interaction (DOI) has been determined using cathode-to-anode ratio and the time difference method based on the calculated mu-tau product [only CAR implemented 11/29]. Energy calibration was performed and resolution was calculated for both the original spectra and the ones applied with electron trapping compensated spectra. Electron trapping compensated spectra demonstrated significantly poorer resolution, suggesting a problem with the detector or methodology. (mutau will be recalculated based on DOI?)
