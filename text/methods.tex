\subsection{Data collection}
Data collection was performed using NRC-exempt check sources of Am-241 and Cs-137. The UCBCZT2 enclosure was placed cathode-side down on a nonmetallic table, and the check source was centered over the detector face (depending on side). A SIS3302 digitizer and USB interface was used to digitize the pulses with 100MHz sampling rate. Countrates exceeding ~100k/min resulted in buffer overflow. In this case, the detector or source was propped up so as to not alter the x/y location, but to increase the z standoff. The trigger lower bound was set to be 8 in the data collection configuration file to properly capture the whole energy spectrum. 

\subsection{Depth of Interaction}
In this lab, we focus on the use of the cathode to anode ratio (CAR) to determine the depth of interaction (DOI) for individual radiation interaction events. As a secondary priority, we investigate the use of event timing to determine this quantity. In the CAR method, the ratio of the charges collected is indicative of the depth, as the weighting potential due to the cathode is approximately linear across the detector (ignoring contributions by holes). To analyze this quanitity, pulses from anodes 1-7 and the cathode (chn 0) were recorded using the SIS3302, shaped using the built-in trapezoid filter and then run through a coincidence detector to match anode and cathode pulses. A coincidence window length of 1us was chosen based on the rise-time properties of the detector.

Subsequent to event-matching, all anode pulses corresponding to a single cathode pulse (resulting from charge-sharing) were amplitude-summed. The cathode-to-anode ratio was then derived by dividing the cathode pulse amplitude by the summed anode value. The resulting values were then put in a scatter plot. Events with CAR values greater than unity are disregarded, as they are the result of spurious coincidence or extremely poor charge collection. Timing readout [not fully implemented as of 11/29] was achieved by reading in the raw pulse data, finding the peak location using the second derivative. Pulse peaking time for anode and cathode was then adjusted for carrier velocity, and the resulting value was used as a depth-of-interaction metric.

Spectra were generated in several ways. A first-pass spectrum was created by simply histogramming all anode pulses. Alternate spectra were generated by histogramming the anode charge values of all pulses which had a coincident cathode pulse, as well as those that had both cathode coincidence pulses and certain CAR values. This process was also repeated for individual pixels in order to address issues due to differing pixel gain.

Spectrum energy correction was performed on an event-by-event basis by using the CAR depth-of-interaction parameter and a charge collection correction factor. This factor was derived using the Hecht relation (see below). This was used to correct all events with valid interaction parameters (0<x<1), while events with invalid interaction parameters were removed from the data set.

\[Q/Q_0 = {\lambda}_e +{\lambda}_h -{\lambda}_e exp(-(1-x)/{\lambda}_e) - {\lambda}_h exp(-x/{\lambda}_h) \]


\subsection{Curve Fitting}
The peaks in the spectra were fit using a gaussian peak, which is a very rough approximation due to the low-energy tailing present in CZT detectors. This fitting was performed using the SciPy curvefit toolkit.




